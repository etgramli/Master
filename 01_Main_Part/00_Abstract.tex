\documentclass[../InterneDSLs.tex]{subfiles}
\begin{document}

\chapter*{Abstract}

\begin{center}
	\begin{tabular}{ll}
		\textbf{Thema:} 		& \topic\\
		\textbf{Verfasser:} 	& \authorName\\
		\textbf{Firma:} 		& \companyName\\
		\textbf{Betreuer:}		& \proffessor\\
								& \carer\\
		\textbf{Abgabedatum:}	& \closingdate\\
	\end{tabular}
\end{center}

\bigskip
Domänenspezifische Sprachen sind ein wichtiger Bestandteil der Softwareentwicklung und sind an einigen Stellen anzutreffen (z.B. Buildsysteme, SQL, \LaTeX). Normalerweise wird schon beim Entwurf der Sprache zwischen externer und interner DSL unterschieden und sich im Falle einer internen DSL auf eine Hostsprache festgelegt.

In dieser Arbeit wird ein Ansatz dargestellt, wie man interne DSLs beschreiben kann, die für mehrere Hostsprachen generiert werden können. Für die agile Softwareentwicklung kann es hilfreich sein, wenn eine interne DSL für mehrere Hostsprachen verfügbar ist, etwa wenn in einem Projekt eine Sprache zu einer anderen migriert wird (z.B. bei Android der Wechsel von Java zu Kotlin). Die Umsetzbarkeit wird an einem Prototypen demonstriert.
\newpage

\end{document}
