\documentclass[../find-a-part.tex]{subfiles}
\begin{document}

\chapter*{Abstract}
\addcontentsline{toc}{chapter}{Abstract (Zusammenfassung)}

\begin{center}
	\begin{tabular}{ll}
		\textbf{Thema:} 		& \topic\\
		\textbf{Verfasser:} 	& \authorName\\
		\textbf{Firma:} 		& \companyName\\
		\textbf{Betreuer:}		& \proffessor\\
								& \carer\\
		\textbf{Abgabedatum:}	& \closingdate\\
	\end{tabular}
\end{center}

\bigskip
Das Warten von Maschinen wird mit deren Komplexität immer schwieriger und die Vielfalt der Einzelteile steigt. Um die Arbeiten an den eigenen Maschinen zu erleichtern, müssen neue Prozesse und Werkzeuge geschaffen werden.

Um Bestandteile eigener Maschinen leicht identifizieren und Ersatzteile nachbestellen zu können, soll eine mobile Applikation entwickelt werden. Ziel ist es mit Teilbildern von der Maschine, auf denen ihre Bestandteile zu sehen sind, deren Komponenten identifizieren können. Das wird durch Abgleichen von Eigenschaften der Bilder (Features) in einer Datenbank realisiert. Weitergehend sollen danach Vorschläge in einem Online-Shop, nach Übereinstimmung sortiert, angezeigt werden.

Bisherige mobile Anwendungen der Bilderkennung taggen die Bilder mit allgemeinen Bezeichnungen, wie \gqq{Person} oder \gqq{Zahnrad}. Das Ziel dieser Arbeit ist es, spezifische Objekte wiederzufinden, statt die grobe Kategorie den Bildern zuzuordnen.
\newpage

\end{document}
