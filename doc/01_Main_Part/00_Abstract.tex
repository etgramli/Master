\documentclass[../find-a-part.tex]{subfiles}
\begin{document}

\chapter*{Abstract}
\addcontentsline{toc}{chapter}{Abstract (Zusammenfassung)}

\begin{center}
	\begin{tabular}{ll}
		\textbf{Thema:} 		& \topic\\
		\textbf{Verfasser:} 	& \authorName\\
		\textbf{Firma:} 		& \companyName\\
		\textbf{Betreuer:}		& \proffessor\\
								& \carer\\
		\textbf{Abgabedatum:}	& \closingdate\\
	\end{tabular}
\end{center}

\bigskip
Domain Specifi Languages (DSLs) are used to express only the problems of a specific domain and no others, unlike General Purpose Languages (GPLs) such as Java, Scala and C++. Thus such a language is easier to learn for a domain expert than a GPL.

There are internal and external DSLs, of which external DSLs require its own tools to use. So for an agile environment an internal DSL is favorable because the tools and pipeline already exist and so the language itself can change more easily.

To easily generate a DSL from an EBNF grammar, a tool should be implemented to generate an internal DSL in a certaini host language. The suggested host language for this tool is Java, as it is widely adopted. If shortcomings of Java are detected, Scala may another good choice, as it is a more flexible language and also runs on the JVM.
\newpage

\end{document}
