\providecommand{\main}{..}		% Override relative path to the main file (already set in main file)
\documentclass[../InterneDSLs.tex]{subfiles}

\begin{document}

\chapter{Einleitung}


\section{Motivation}
Domänenspezifische Sprachen sind ein zunehmend wichtiger werdender Trend in der Softwareentwicklung. Durch domänenspezifische Sprachen können Sachverhalte einer Domäne einfach ausgedrückt und daraus Programme erstellt werden. 

Da im agilen Umfeld ein schneller Einstieg in die Entwicklung wichtig ist, bieten sich hier interne \acsp{DSL} an, weil es für die \Gls{Hostsprache} bereits Tools (z.B. Compiler, Editor) gibt, die nicht neu erstellt werden müssen.


\section{Zielsetzung}
Das Ziel der Arbeit ist es zu untersuchen, welche Arten von Grammatiken (regulär, \ac{EBNF}, etc.) sich als interne DSL umsetzen lassen. Als Hostsprache ist Java ins Auge gefasst, da Java weit verbreitet ist, auch wenn Java nicht die flexibelste Sprache ist. Sollte während der Arbeit an die Grenzen von Java gestoßen werden, können auch andere \ac{JVM}-Sprachen (etwa Scala) betrachtet werden.


\section{Aufbau der Arbeit}



\end{document}
