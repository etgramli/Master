\providecommand{\main}{..}		% Override relative path to the main file (already set in main file)
\documentclass[../InterneDSLs.tex]{subfiles}
\begin{document}

\section{Hauptteil}
Für die Verarbeitung der \ac{EBNF}-Grammatiken wird ANTLR4 verwendet, der Parser und Lexer für Java generieren kann.

\subsection{Parser}
Als Parser-Generator wurde ANTLR4 verwendet, der Parser zu einer EBNF-Grammatik in Java generieren kann. Dafür verwendet ANTLR aber eine unübliche Syntax, die sich an einer alten EBNF-Variante orientiert.

Der von ANTLR generierte Parser wird von einem Listener erweitert, der aus den EBNF-Konstrukten (die praktisch Strings entprechen) einen Baum aus eigenen Klassen generiert. Dadurch wird eine Typ-Sicherheit hergestellt, die durch die EBNF selbst nicht gegeben ist.

\subsection{Abbildung von EBNF auf Interfaces}
Zuerst wird die in EBNF geschriebene Grammatik in einen Baum aus Java-Objekten transformiert. Dadurch wird eine Typ-Sicherheit hergestellt und Methoden implementiert werden, die das Traversieren des Baumes vereifachen. Abbildung~\ref{FIG:TypesBNF} zeigt die Typen und ihre Beziehungen untereinander.

\begin{figure}
\centering
\includegraphics[width=\linewidth]{\main/10_Pictures/BNF-Types.eps}
\caption{Typen des BNF-Baumes}
\label{FIG:TypesBNF}
\end{figure}

\subsubsection{Parse-Tree}

\subsection{Sprachdesign}


\subsubsection{Einschränkungen}


\end{document}
