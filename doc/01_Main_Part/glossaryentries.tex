% Glossaries to be used from 'glossaries' package

\newglossaryentry{.NET}{%
  name={.NET},
  description={Eine Software-Plattform von Microsoft, die mehrere Programmiersprachen und eine Laufzeitumgebung beinhaltet. Das ganze .NET-Framework läuft nur unter Windows, und eine Teilmenge (.NET-Core) läuft auch unter Linux und Mac OS X}
}

\newglossaryentry{Amazon Web Services}{%
  name={Amazon Web Services},
  description={Ein flexibler Cloud-Computing-Service von Amazon mit verschiedenen Kategorien: Virtuelle Server, Speicher, Datenbanken und anderem}
}

\newglossaryentry{GitHub}{%
  name={GitHub},
  description={Eine Plattform auf der man Projekte hosten kann, die mit Git verwaltet werden}
}

\newglossaryentry{Hosting}{%
  name={Hosting},
  description={Ein Dienst, bei dem der Dienstleister die Infrastruktur für den Betrieb eines Services im Internet bereitstellt}
}

\newglossaryentry{Java Server Page}{%
  name={Java Server Page},
  description={Eine Programmiersprache von Sun, um dynamisch HTML zu erzeugen}
}

\newglossaryentry{Java Virtual Machine}{%
  name={Java Virtual Machine},
  description={Die Laufzeitumgebung der Sprache Java; es gibt verschiedene Implementierungen für unterschiedliche Plattformen. Darunter sind Windows, Linux Mac OS X, Android und Solaris}
}

\newglossaryentry{KAZE}{%
  name={KAZE},
  description={Ein Feature-Deskriptor, der nach dem japanischen Wort für Wind benannt wurde.}
}

\newglossaryentry{Keypoint}{%
  name={Keypoint},
  description={Eine auffällige Stelle in einem Bild, wie eine Ecke oder Kante, die von einem Filter gefunden wird}
}

\newglossaryentry{K-Nearest Neighbors}{%
  name={K-Nearest Neighbors},
  description={Eine Methode zur Klassifikation, das die k nächsten Objekte aus einer Menge findet}
}

\newglossaryentry{Lucene}{%
  name={Lucene},
  description={Eine Bibliothek von der Apache Software Foundation zur Volltextsuche veröffentlicht unter der Apache-Lizenz}
}

\newglossaryentry{MPEG-7}{%
  name={MPEG-7},
  description={Ein ISO-Standard von 2002 der \ac{MPEG}, die Descriptors, Description Schemes und eine \ac{DDL} definiert, um Audio- oder Video-Inhalte mit (inhaltlichen und semantischen) Metadaten zu versehen}
}

\newglossaryentry{Port}{%
  name={Port},
  description={Übertragung einer Software auf eine andere Plattform}
}

\newglossaryentry{Query}{%
  name={Query},
  plural={Queries},
  description={Anfrage an ein bestimmtes System}
}

\newglossaryentry{Representational State Transfer}{%
  name={Representational State Transfer},
  description={Eine Eigenschaft von Webservices, bei dem man, mit einer definierten und zustandslosen Operation, eine Ressource verändern kann. Die Operationen sind zur Maschine-zu-Maschine-Kommunikation im Client-Server-Modell gedacht}
}

\newglossaryentry{Semantische Lücke}{%
  name={Semantische Lücke},
  plural={Semantische Lücken},
  description={Damit wird der Sachverhalt bezeichnet, dass eine maschinelle oder logische Beschreibung einer Sache nicht alle Aspekte abdeckt, die für einen Nutzer oder das reale Leben relevant sind}
}

\newglossaryentry{Servlet}{%
  name={Servlet},
  description={Klassen, die dynamisch von einem Webserver instantiiert werden, um Anfragen zu bearbeiten}
}

\newglossaryentry{Simple Object Access Protocol}{%
  name={Simple Object Access Protocol},
  description={Das Simple Object Access Protocol ist ein W3C-Standard zum Übertragen von Daten und Ausführen von Remote Procedure Calls, der meistens mit HTTP und XML umgesetzt wird}
}

\newglossaryentry{Tagging}{%
  name={Tagging},
  description={Das Versehen eines Objektes mit einem aussagekräftigen Wort}
}

\newglossaryentry{Tag}{%
  name={Tag},
  description={Ein aussagekräftiges Wort, das als Meta-Information einem Objekt zugeordnet wird}
}
