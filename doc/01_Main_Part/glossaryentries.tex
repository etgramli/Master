% Glossaries to be used from 'glossaries' package

\newglossaryentry{ANTLR4}{%
  name={ANTLR 4},
  description={Ein in Java geschriebener Parsergenerator, der verschiedene Zeilsprachen unterstützt (z.B. Java, C++, C\#, Python). Es werden auch LL(*)-Grammatiken unterstützt}
}

\newglossaryentry{GitHub}{%
  name={GitHub},
  description={Eine Plattform auf der man Projekte hosten kann, die mit Git verwaltet werden}
}

\newglossaryentry{Hostsprache}{%
  name={Hostsprache},
  description={Eine General-Purpose Programmiersprache, in die eine interne DSL eingebettet ist}
}

\newglossaryentry{Java Virtual Machine}{%
  name={Java Virtual Machine},
  description={Die Laufzeitumgebung der Sprache Java; es gibt verschiedene Implementierungen für unterschiedliche Plattformen. Darunter sind Windows, Linux Mac OS X, Android und Solaris}
}

\newglossaryentry{Kontextfreie Grammatik}{%
  name={Kontextfreie Grammatik},
  description={Kontextfreie Grammatiken (\ac{CFG}) sind identisch mit den Typ-2-Grammatiken. Es folgt auf eine Folge von Terminal- und Nichtterminalsymbolen immer ein Nichtterminalsymbol}
}

\newglossaryentry{Reguläre Grammatik}{%
  name={Reguläre Grammatik},
  description={Reguläre Grammatiken sind identisch mit den Typ-3-Grammatiken der Chomsky-Hierarchie. Es sind kontextfreie Grammatiken, die weiteren Einschränkungen unterliegen. Die linke Seite der Regel darf nur ein nichtterminales Symbol beinhalten, die Rechte Seite höchstens ein Nichtterminales.}
}

\newglossaryentry{Semantische Lücke}{%
  name={Semantische Lücke},
  plural={Semantische Lücken},
  description={Damit wird der Sachverhalt bezeichnet, dass eine maschinelle oder logische Beschreibung einer Sache nicht alle Aspekte abdeckt, die für einen Nutzer oder das reale Leben relevant sind}
}

\newglossaryentry{Tagging}{%
  name={Tagging},
  description={Das Versehen eines Objektes mit einem aussagekräftigen Wort}
}

\newglossaryentry{Tag}{%
  name={Tag},
  description={Ein aussagekräftiges Wort, das als Meta-Information einem Objekt zugeordnet wird}
}
