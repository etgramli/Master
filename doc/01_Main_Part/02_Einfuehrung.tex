\providecommand{\main}{..}		% Override relative path to the main file (already set in main file)
\documentclass[../InterneDSLs.tex]{subfiles}
\begin{document}

\section{Einführung}

\subsection{Grammatiken}
Programmiersprachen liegen formale Sprachen zugrunde, die sich in der Chomsky-Hie"-rar"-chie (Abbildung~\ref{FIG:Chomksky-Hierarchie}) wiederfinden. Diese formalen Sprachen legen durch Regeln fest, welche Zeichenketten gültig sind und damit zur Sprache gehören. Jede Klasse von Grammatiken beinhaltet auch die Grammatiken tiefer in der Chomsky-Hierarchie. (Überführung ineinander?)

\begin{figure}
\centering
\def\svgwidth{0.5\textwidth}
\input{\main/10_Pictures/Grammatiken.eps_tex}
\caption{Chomsky-Hierarchie}
\label{FIG:Chomksky-Hierarchie}
\end{figure}

\subsubsection{Typ-3-Grammatiken}
Reguläre Grammatik

\subsubsection{Typ-2-Grammatiken}
Kontextfrei

\subsubsection{Typ-1-Grammatiken}
Kontextsensitiv

\subsubsection{Typ-0-Grammatiken}
Beliebige formale Grammatik


\subsection{Baumerzeugung}


\subsubsection{AST}


\subsubsection{Parse-Tree}



\subsection{Parser Types}


\subsubsection{LL}


\subsubsection{LR}


\subsubsection{Parsing Patterns}


\end{document}
